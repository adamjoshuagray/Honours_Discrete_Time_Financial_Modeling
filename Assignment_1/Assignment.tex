\documentclass{unswmaths}
\usepackage[a4paper]{geometry}
\usepackage{fancyhdr}
\usepackage{hyperref}

\pagestyle{fancy}
\begin{document}

\setlength\parindent{0pt}

\unswtitle{Adam J. Gray}{3329798}{Discrete Time Financial Modeling}{Assignment}
\fancyfoot[l]{Adam J. Gray}
\fancyfoot[r]{\today}
\fancyhead[l]{The University of New South Wales}
\fancyhead[r]{Discrete Time Financial Modeling}

\section*{Question 1}
Consider the model with two states of the world, one risky security $ S_n, n= 1 $
and one taking the interest rate $ r = \frac{1}{9} $, along with the following security prices.
\begin{table}[h]
	\centering
	\begin{tabular}{|l|c|c|c|}
		\hline
		$ n $ & $ S_n(0) $ & $ S_n(1)(\omega_1) $ & $ S_n(1)(\omega_2) $ \\
		\hline
		$ 1 $ & $ 5 $ & $ \frac{20}{3} $ & $ \frac{40}{9} $ \\
		\hline
	\end{tabular}
\end{table}
What is the fair price of a European put option with strike price $ K = 5$?
What trading trategy generates this contingent claim?

\subsection*{Solution}
In general, for a one step model,
\begin{align*}
	V_{put} = \frac{1}{1+r} \left[ \frac{1+r-d}{u-d} \left( K - S_0 u \right)^+ 
		+ \frac{u-r-1}{u-d} \left( K - S_0 d \right)^+ \right]
\end{align*}
and in this case we have $ K = 5 $, $ u = \frac{4}{3} $, $ d = \frac{8}{9} $ $, S_0 = 5 $, $ r = \frac{1}{9} $ so
\begin{align*}
	V_{put} &= \frac{1}{1+\frac{1}{9}} \left[ \frac{1+\frac{1}{9}-\frac{8}{9}}{\frac{4}{3}-\frac{8}{9}} \left( 5 - 5 \frac{4}{3} \right)^+ 
		+ \frac{\frac{4}{3}-\frac{1}{9}-1}{\frac{4}{3}-\frac{8}{9}} \left( 5 - 5 \frac{8}{9} \right)^+ \right] \\
		&= \frac{9}{10}\left[ 0 + \frac{1}{2} \times \frac{5}{9} \right] \\
		&= \frac{1}{4}
\end{align*}
This pricing corresponds to the following trading strategy:
\subsubsection*{at time 0}
Sell 1 put for $ \frac{1}{4} $. \hfill Cash Flow: $+\frac{1}{4}$ 

Sell $ \frac{1}{4} $ shares. \hfill Cash Flow: $+\frac{5}{4} $

Invest $ \frac{3}{2} $ at the risk free rate. \hfill Cash Flow: $-\frac{3}{2} $

\hfill Total: $ 0 $

\subsubsection*{at time 1 if $ S_1 = \frac{20}{3} $}
Payoff option. \hfill Cashflow: $0$

Liquidate risk free rate investment. \hfill Cashflow: $+\frac{5}{3}$

Repurcahse shares. \hfill Cashflow: $-\frac{5}{3}$

\hfill Total: $ 0 $

\subsubsection*{at time 1 if $ S_1 = \frac{40}{9} $}
Payoff option. \hfill Cashflow: $-\frac{5}{9}$

Liquidate risk free rate investment. \hfill Cashflow: $+\frac{5}{3} $

Repurchase shares. \hfill Cashflow: $-\frac{10}{9}$

\hfill Total: $0$

\section*{Question 2}
Suppose the interest rate $r$ is a scalar, and let $c$ and $p$ denote the prices of a call and
put, respectively, both having the same strike price $K$. Show that either both are
attainable or neither is attainable. Use risk neutral valuation to show that in the
former case one has
$$
    c - p = S_0 - \frac{K}{1+r}.
$$

Assume the put is attainable. We wish to show that the call is attainable, i.e that using a put
we can replicate the payoff of the call.

\subsubsection*{Solution}
Assume the put is attainable. We wish to show that the call is attainable, i.e that using a put
we can replicate the payoff of the call.

\subsubsection*{time 0}
Purchase 1 put. \hfill Cashflow: $-p $

Purchase 1 share. \hfill Cashflow: $-S_0$

Borrow $ \frac{K}{1+r} $ at the risk free rate. \hfill Cashflow: $+ \frac{K}{1+r}$

\subsubsection*{time 1 when $ S_1 = S_1(up) $}
Collect payoff from the put.    \hfill Cashflow: $ (K-S_1(up))^+ $

Repay loan.     \hfill Cashflow: $ - K $

Sell share.     \hfill Cashflow: $ S_1(up) $

\hfill Total: $ S_1(up) - K + (K - S_1(up))^+ = (S_1(up)-k)^+$

\subsubsection*{time 1 when $ S_1 = S_1(down) $}
Collect payoff from the put.    \hfill Cashflow: $ (K-S_1(down))^+ $

Repay loan.     \hfill Cashflow: $ - K $

Sell share.     \hfill Cashflow: $ S_1(down) $

\hfill Total: $ S_1(down) - K + (K - S_1(down))^+ = (S_1(down)-k)^+$

Notice that in either case this is just the payoff of a call option. So if the put
is attainable the call is attainable.


Assume that the call is attainable.
\subsubsection*{time 0}
Purcahse 1 call. \hfill Cashflow $ -c $

Sell 1 share.   \hfill Cashflow $ -S_0$

Invest $ \frac{K}{1+r} $ at the risk free rate. \hfill Cashflow: $ -\frac{K}{1+r} $

\subsubsection*{time 1 when $ S_1 = S_1(up) $}
Collect payoff from the call.    \hfill Cashflow: $ (S_1(up)-K)^+ $

Liquidate risk free investment. \hfill Cashflow: $ K $

Repurcahse the share.     \hfill Cashflow: $ -S_1(up) $
\hfill Total: $ (S_1(up)-K)^+ + K -S_1(up) = (K - S_1(up))^+ $

\subsubsection*{time 1 when $ S_1 = S_1(down) $}
Collect payoff from the call.    \hfill Cashflow: $ (S_1(down)-K)^+ $

Liquidate risk free investment. \hfill Cashflow: $ K $

Repurcahse the share.     \hfill Cashflow: $ -S_1(down) $
\hfill Total: $ (S_1(down)-K)^+ + K -S_1(down) = (K - S_1(down))^+ $

So in either case we just have the payoff of a put option. So if the call is attainable,
then the put is. 

This is to say that either the put \emph{and} call are attainable, or neither are.

This demonstration hints that we should have $ c = p + S_0 - \frac{K}{1+r} $ because the
if two portfolios have the same future payoffs that must have the same current value. The
following risk neutral valuation formalises this. 

Observe that
$$
    (S_1 - K)^+ - (K-S_1^+ = S_1 - K
$$
which is equivelent to
$$
    \frac{(S_1 - K)^+ + K}{1+r} = \frac{(K - S_1)^+ + S_1}{1+r}.
$$

Now taking conditional expectations of each side with respect to the risk neutral measure yields
\begin{align*}
    \mathbb{E}_{\mathbb{P}}\left[(1+r)^{-1}(S_1 - K)^+ | S_0 \right] + \frac{K}{1+r}
    &=
    \mathbb{E}_{\mathbb{P}}\left[(1+r)^{-1}(K - S_1)^+ | S_0 \right] + \mathbb{E}_{\mathbb{P}}\left[\frac{S_1}{1+r} | S_0 \right]
\end{align*}
Now using the fact that under the risk free measure the discounted value of a risky asset 
follow a martingale process we must have
\begin{align*}
    \underbrace{\mathbb{E}_{\mathbb{P}}\left[(1+r)^{-1}(S_1 - K)^+ | S_0 \right]}_{\text{ value of call option at time 0 } } + \frac{K}{1+r}
    &=
    c + \frac{K}{1+r} \\
    \underbrace{\mathbb{E}_{\mathbb{P}}\left[(1+r)^{-1}(K - S_1)^+  | S_0 \right]}_{ \text{value of put option at time 0} } + \underbrace{\mathbb{E}_{\mathbb{P}}\left[\frac{S_1}{1+r} | S_0 \right]}_{ \text{value of stock at time at time 0}}
    &=
    p + S_0
\end{align*}
which leaves us with \footnote{ This proof was modeled off a proof by Ophir Gottlieb available at \url{http://www.soarcorp.com/research/put_call_parity.pdf} }
$$
    c = p + S_0 - \frac{K}{1+r}.
$$\qed
\section*{Question 3}
Under the risk neutral measure we have $ \tilde{p} = \tilde{q} = \frac{1}{p} $
and in this case the distribution of $ S_3 $ is
\begin{tabular}[h]{c|cccc}
$ S_3 $ & 32 & 8 & 2 & $ \frac{1}{2} $\\
\hline
$ \mathbb{P} $ & $ \frac{1}{8} $ & $ \frac{3}{8} $ & $ \frac{3}{8} $ & $ \frac{1}{8} $
\end{tabular}

We also have $ \tilde{\mathbb{E}}[S_1] = \frac{8}{2} + \frac{2}{2} = 5 $, $  \tilde{\mathbb{E}}[S_2] = \frac{16}{4} + \frac{8}{4} + \frac{1}{4} = \frac{25}{4} $
and  $ \tilde{\mathbb{E}}[S_3] = \frac{32}{8} + \frac{24}{8} + \frac{6}{8} + \frac{0.5}{8} = \frac{125}{16}$

By taking the geometric mean the average rate of growth is $ \frac{1}{4} $ which is consistent
with the risk free rate implied in $ \tilde{p} = \frac{1+r-d}{u-d} $.

In the case when we have $ p = \frac{2}{3} $, $ q = \frac{1}{3} $ we have

\begin{tabular}[h]{c|cccc}
$ S_3 $ & 32 & 8 & 2 & $ \frac{1}{2} $\\
\hline
$ \mathbb{P} $ & $ \frac{8}{27} $ & $ \frac{12}{27} $ & $ \frac{6}{27} $ & $ \frac{1}{27} $
\end{tabular}

with the following expectations:
$ \mathbb{E}[S_1] = \frac{16}{3} + \frac{2}{3} = 6 $ and $ \mathbb{E}[S_2] = \frac{64}{9} + \frac{16}{9} + \frac{1}{9} = 9 $
and $ \mathbb{E}[S_3] = \frac{256}{27} + \frac{96}{27} + \frac{12}{27} + \frac{0.5}{27} = 13.5 $ 

The average rate of growth is $ 0.5 $.
\end{document}
